\documentclass[12pt,letterpaper]{article}
\usepackage[utf8]{inputenc}

\title{Problems with existing decentralized ecosystems}
\author{David Kerkeslager}
\date{}

\begin{document}

\begin{titlepage}
\maketitle
\end{titlepage}

\section{Second-class citizen problem}
Applications which run on Ethereum's ecosystem pay a tax to Ethereum in the
form of gas. Applications provide value, and bring users and money into
Ethereum's ecosystem. As a result, the value of Ethereum rises, which in turn
means that the tax which applications pay to Ethereum rises. At the time of
this writing, many applications are leaving the Ethereum ecosystem due to
these rising gas fees. In short, Ethereum builds its success upon apps, and
when Ethereum succeeds, it punishes the very apps upon which its success was
built.

\section{Gas problem}
When a program runs out of gas, it is reverted; the person sending the program
pays nothing. Programs which halt on time, on the other hand, pay dearly for
the chance to run. In this way, programs which behave properly are punished,
while programs which misbehave are allowed to go free.

\end{document}
