\documentclass[12pt,letterpaper]{article}
\usepackage[utf8]{inputenc}

\title{Requirements for a digital currency}
\author{David Kerkeslager}
\date{}

\begin{document}

\begin{titlepage}
\maketitle
\end{titlepage}

\section{Inflation as a necessary component of currency}

\section{Inclusion fees considered harmful}
In both proof of stake and proof of work, the provers are incentivized
with inclusion fees. However, as the value of the currency rises and the
network becomes congested, these fees rise to the point where the
cryptocurrency becomes unusable as a currency. Once this process is started
it cannot be stopped, because provers are incentivized to refuse to lower
inclusion fees. This process continues until provers kill the golden goose.
In the case of proof of work, provers (miners) simply move to another
blockchain, as they have no investment in the coin itself. In the case of proof
of stake, provers (stakers) attempt to settle at a level of fees such that fee
\times inclusions is maximized: too high a fee and fewer users will submit
inclusions, while if the fee is any lower, stakers are under-profiting. While
this has a certain efficency to it, it results in a system which serves provers
rather than users.

\end{document}
